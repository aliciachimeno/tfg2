\documentclass[../main.tex]{subfiles}

\begin{document}

\appendix

\chapter{ Theory used} \label{apx:purification_with_noise}

\begin{definition} Riemann integral reminder. 
The Riemann integral is a method for calculating the volume under a curve of a continuous function on a closed, bounded domain in $R^n$. The method involves dividing the domain into smaller subregions and approximating the volume of each subregion with a rectangular solid whose height is the function value at a specific point in the subregion. The Riemann sum is the sum of the volumes of all the rectangular solids, and as the size of the subregions approaches zero, the Riemann sum converges to the Riemann integral. \\ \\
\end{definition}

\begin{definition}
	Let $\Sigma$ be a $\sigma$-algebra over a set $\Omega$. A \emph{measure} over $\Omega$ is any function $$\mu:\Sigma\longrightarrow[0,\infty]$$ satisfying the following properties:
	\begin{enumerate}
		\item $\mu(\varnothing)=0$.
		\item\label{RFA:sigmaadditivity} \emph{$\sigma$-additivity}: If $(A_n)\in\Sigma$ are pairwise disjoint, then: $$\mu\left(\bigsqcup_{n=1}^\infty A_n\right)=\sum_{n=1}^\infty \mu(A_n)$$
	\end{enumerate}
\end{definition}

\begin{definition}The closure of a set A of a metric space (X, d) is defined as follows:
	$$closure(A) = \overline{A}=\{t \, |  \, \forall \epsilon> 0, \exists a \in A, \, d(a, t) < \epsilon\}.$$
\end{definition}

\begin{propo}
 Let $(X,\tau)$ be a topological space and $A\subseteq X$ be a subset. Then, $A$ is dense in $(X,\tau)$ if and only if $\overline{A}=X$.
\end{propo}

\begin{definition} A metric space $(X,d)$ is said to be \emph{complete} if every Cauchy sequence in $X$ converges to a point in $X$. 
\end{definition}

\begin{definition}
We say that a property holds almost everywhere (a.e.) if the set of points that doesn’t hold it is null.
\end{definition}


\begin{definition} 
	$\varphi$ : $I \rightarrow \mathbb{R}$ is uniformly continuous on $I$ if $\forall \epsilon > 0 \exists \delta >0 $ such that $|\varphi(x)- \varphi(y)| < \epsilon$ whenever $|x-y|< \delta$
\end{definition}


\section{Blaire's category theorem}

\begin{definition} Let $A$ be a subset of the metric space $(X,d)$. $A$ is said to be \emph{nowhere dense} if for every (nonempty) open subset $U\subseteq X$, the intersection $U\cap\overline{A}$ is not dense in $U$, meaning that $U$ contains a point that is not in the closure of $A$.
\end{definition}
\begin{definition}
	A set it is said to be category $I$ if it can be written as a countable union of nowhere-dense sets. Otherwise it is said to be of \emph{category $II$}
\end{definition}
\begin{theorem} (Blaire's Category Theorem)
	Any complete metric space is of category $II$.
\end{theorem}
\noindent Therefore, if we have $\mathcal{C}_0^\infty[a,b]$ complete metric space, we know that is of category $II$, i.e. $\mathcal{C}_0^\infty[a,b]$  cannot be written as a countable union of nowhere-dense sets. We have $\cup_{k=0}^\infty V_k = \mathcal{C}_0^\infty[a,b]$. Therefore, some $V_m$ contains a nonempty open set. $V_m$ is a vector space thus $V_m=\mathcal{C}_0^\infty[a,b]$. no entenc el final
\end{document}