%%TC:ignore
\documentclass[../../main.tex]{subfiles}

\begin{document}



\section*{Abstract} % starred section doesn't get the correct header


\noindent Nowadays, machine learning models are being applied more and more, but it is the task of mathematicians to understand their complex underlying principles. How can we ensure the existence of a predictive function from a given dataset? In this work, we will take an analytical approach to machine learning, emphasizing function approximation as a central component. This research seeks to address these concerns by exploring the mathematical foundations of function approximation in machine learning, with a specific focus on neural networks.
\\ \\
In particular, we delve into a significant finding, the theorem proved by Leshno-Lin-Pinkus-Schonken in 1993 \cite{leshno1993multilayer}, which states that a multilayer feedforward network equipped with a non-polynomial activation function can effectively approximate any continuous function. Our work revolves around understanding and reinterpreting the proof, while expanding and providing further details.
Through this study, we aim to bridge the gap between the practical application of machine learning and the mathematical principles that underpin its success.

\newpage

\section*{Resum}
En l'actualitat, cada cop més s'apliquen models de machine learning, però és tasca dels matemàtics entendre el seu complex rerefons. Com podem garantir l'existència d'una funció predictiva a partir d'un conjunt de dades donat?  En aquest treball prendrem una visió analitica al machine learning posant èmfasi en l'aproximació de funcions com a component central. Aquesta recerca pretén abordar aquestes qüestions, explorant els fonaments matemàtics de l'aproximació de funcions en l'aprenentatge automàtic, amb un focus específic en les xarxes neuronals.

En particular, aprofundim en una troballa important, el teorema demostrat per Leshno-Lin-Pinkus-Schonken el 1993, que afirma que una xarxa d'alimentació cap endavant multicapa equipada amb una funció d'activació no polinomial pot aproximar efectivament qualsevol funció contínua. El nostre treball gira entorn de comprendre i reinterpretar la demostració, alhora que s'amplia i es proporcionen més detalls.
A través d'aquest estudi, pretenem establir un nexe entre l'aplicació pràctica de l'aprenentatge automàtic i els principis matemàtics que sustenten el seu èxit.
\newpage
\section*{Preface}



blslalblallb en català
\end{document}
%%TC:endignore