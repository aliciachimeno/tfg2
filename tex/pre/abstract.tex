%%TC:ignore
\documentclass[../../main.tex]{subfiles}

\begin{document}



\section*{Abstract} % starred section doesn't get the correct header


\noindent Nowadays, machine learning models are being applied more and more, but it is the task of mathematicians to understand their complex underlying principles. How can we ensure the existence of a predictive function from a given dataset? In this work, we will take an analytical approach to machine learning, emphasizing function approximation as a central component. This research seeks to address these concerns by exploring the mathematical foundations of function approximation in machine learning, with a specific focus on neural networks.
\\ \\
In particular, we delve into a significant finding, the theorem proved by Leshno-Lin-Pinkus-Schonken in 1993 \cite{leshno1993multilayer}, which states that a multilayer feedforward network equipped with a non-polynomial activation function can effectively approximate any continuous function. Our work revolves around understanding and reinterpreting the proof, while expanding and providing further details.
Through this study, we aim to bridge the gap between the practical application of machine learning and the mathematical principles that underpin its success.

\newpage

\section*{Resum}
\noindent En l'actualitat, cada cop s'apliquen més i més els models de machine learning, però és tasca dels matemàtics entendre el seu complex rerefons. Com podem garantir l'existència d'una funció predictiva a partir d'un conjunt de dades donat?  En aquest treball prendrem una visió analitica del machine learning posant èmfasi en l'aproximació de funcions com a component central. Aquesta recerca pretén abordar aquestes qüestions, explorant els fonaments matemàtics de l'aproximació de funcions en l'aprenentatge automàtic amb un focus específic en les xarxes neuronals.
\\ \\
\noindent En particular, aprofundim en una troballa important, el teorema demostrat per Leshno-Lin-Pinkus-Schonken el 1993, que afirma que una xarxa neuronal equipada amb una funció d'activació no polinomial pot aproximar qualsevol funció contínua. El nostre treball gira entorn a comprendre i reinterpretar la demostració, alhora que ampliar i proporcionar més detalls.
A través d'aquest estudi, pretenem establir un nexe entre l'aplicació pràctica de l'aprenentatge automàtic i els principis matemàtics que sustenten el seu èxit.
\newpage

\section*{Acknowledgments}
\noindent This work marks the end of a beautiful chapter in my life. Since high school, I have been passionate about mathematics. It was the only subject where I truly enjoyed studying, and even though I knew all the exercises on Euclidean plane by heart, I would repeat them over and over again. However, I always had uncertainty about which career path to choose. That is why I want to thank my high school mathematics teachers Javier Ibañez, Jordi Sorolla and Jordi Carrillo for encouraging me and instilling their passion for numbers in me. Without them, I wouldn't be here today.\\ \\
\noindent This degree has been an exciting journey filled with effort and sacrifice, where I have been able to indulge in my love for studying mathematics alongside incredible and highly intelligent people. I am also grateful to all the professors who have taught me throughout my time at the university. And those who dazzle with passion when explaining, I will always remember them. I would also like to express my heartfelt gratitude to my supervisor, Roberto Rubio. Thank you for guiding and assisting me throughout the development of this thesis and for sharing your passion for mathematics with me.\\ \\
\noindent Finally, I would like to thank my parents and my brother for supporting me through the highs and lows and for their unconditional love.



\end{document}
%%TC:endignore