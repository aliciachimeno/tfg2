\documentclass[../main.tex]{subfiles}
\begin{document}
\chapter{Function Approximation} \label{ch:methods}
\noindent Creating a machine learning model to predict/classify from a given data is a similar process than when we calculate a function from a given points in the space. This is called function approximation and among the most famous techniques of function approximation, we find interpolation: such as Taylor polynomial, Chebyshev polynomial, the method of least squares, or spline approximation. In this chapter we are going to talk about ...
\section{????}
\noindent In this section we present some mathematical definitions and results of function approximation. 
If we want to approximate functions, we need to define the following notions: distance between functions, density,


\begin{definition}\label{thm:first}
	A \emph{metric} (or \emph{distance}) on a set $X$ is a function $d:X\times X\rightarrow \mathbb{R} $ such that for all $s,t, u\in X$ the following properties are satisfied:
	\begin{enumerate}
		\item $d(s,t) \geq 0$ and $ \, d(s,t)=0$ if and only if $ s=t$.
		\item $d(s,t)=d(t,s)$.
		\item $d(s,t)\leq d(s,u)+d(u,t)\quad$(\emph{triangle inequality}).
	\end{enumerate}
A \emph{metric space} is a pair $(X,d)$, where $X$ is a set and $d$ is a distance in $X$.
\end{definition}
\noindent  If we take $X$ to be a set of functions, the metric $d(f,g)$ will enable us to measure the distance between functions $f,g \in X$.


\begin{definition} We dentoe by $\mathcal{C}(\mathbb{R}^n)$ the set of continuous functions defined on $\mathbb{R}^n$.
\end{definition}

\begin{definition} We denote by
	$ \mathcal{C}^\infty_0$ the set of infinitely differentiable functions (also called smooth functions), $\mathcal{C}^\infty$, with compact support. Recall that the support of a function $u$ is denoted by $$supp(u)= \overline{\{x | u(x)\neq 0\}}$$
\end{definition}
\begin{propo} \label{prop:frech}
Let $\rho$ be a metric defined on the set $\mathcal{C}^\infty_0[a,b]$ as follows:
$$\rho(\varphi_1,\varphi_2) = \sum_{n=0}^\infty 2^{-n}  \frac{\|\varphi_1 -\varphi_2\|_n}{1+\|\varphi_1 -\varphi_2\|_n}$$ where $$\|\varphi\|_n= \sum_{j=0}^n \sup_{x\in [a,b]} | \varphi^{(j)}(x)|. $$  Then the metric space $(\mathcal{C}^\infty_0[a,b],\rho)$  is complete, also known as a Fréchet space.
\end{propo}



\subsection{Lebesgue measure}

\begin{definition} A box in $\mathbb{R}^d$ is a set of the form 
	$$ Q = [a_1,b_1]  \times ... \times [a_d,b_d] = \prod_{i=1}^d [a_i,b_i]$$
	The volume of the box is $$ vol(Q)= (b_1,a_1) ... (b_d-a_d)= \prod_{i=1}^d (b_i-a_i)$$
	The \emph{exterior measure} (or outer measure) of a set $E\subseteq \mathbb{R}^d$ is $$ |E|^* = \inf\{\sum_k vol(Q_k)\} $$ 
	where the infimum is taken over all finite or countable collection of boxes $\{Q_k\}$ such that $E \subseteq \cup_k Q_k$
\end{definition}

\begin{definition}
A set $E\subseteq \mathbb{R}^n $ is \emph{Lebesgue mesurable} (or mesurable) if $\forall \epsilon >0$, there exist $U$ open set such that $E \subseteq U$ and $|U\setminus E|^* < \epsilon $
\end{definition}

\begin{definition}
	We say that a property holds almost everywhere (a.e.) if the set of points that doesn’t hold it is null.
\end{definition}
\begin{definition}
	A function $u$ defined almost everywhere on a measurable set $\Omega \in \mathbb{R}^n$ is said to be \emph{essentially bounded} on $\Omega$ if $|u(x)|$ is bounded almost everywhere on $\Omega$. We denote $u\in L^{\infty}(\Omega)$ with the norm $$\|u\|_{L^{\infty}(\Omega)}= inf(\lambda | \{ x : |u(x)| \geq \lambda \} = 0 ) = ess \sup_{x\in \Omega} |u(x)|$$

\end{definition}

\noindent We have that $L^{\infty}(\mathbb{R})$ is the space of essentially bounded functions.


\begin{definition}
	A function u defined almost everywhere on a domain $\Omega$ (a domain is an open set in $\mathbb{R}^n$) is said to be\emph{ locally essentially bounded }on $\Omega$ if for every compact set $K\subset \Omega$, $u\in L^{\infty}(K)$. We denote $u\in L_{loc}^{\infty}(K)$.
\end{definition}

\begin{definition} Let  $\mathcal{M}$ denote the set of functions which are in $L_{loc}^{\infty}(\mathbb{R})$ and have the following property. The closure of the set of points of discontinuity of any function in $\mathcal{M}$ is of zero Lebesgue measure. 
\end{definition}
\noindent This implies that for any $\sigma \in$ $\mathcal{M}$, interval $[a,b] .$ and $\delta >0$, there exists a finite number of open intervals, the union of which we denote by U, of measure $\delta$, such that $\sigma$ is uniformly continuous on $[a,b]/U$. 


\begin{definition}We say that a set of functions $F\subset L_{loc}^{\infty}(\mathbb{R})$ is \emph{dense} in $C(\mathbb{R}^n)$ if for every function $g\in C(\mathbb{R}^n)$ and for every compact $K\subset \mathbb{R}^n$, there exist a sequence of functions $f_j\in F$ such that $$\lim_{j\rightarrow\infty} \|g-f_j\|_{L^\infty(K)}=0.$$ 
\end{definition}

\subsection{Convolution}
\begin{definition}
	Let $f,g$ be real-valued functions with compact support. We define the \emph{convolution} of $f$ with $g$ as $$(f\ast g)(x)=\int f(x-t)g(t) \, dt$$
\end{definition}


\begin{propo}
	If $f$ is a smooth function that is compactly supported and $g$ is a distribution, then $f\ast g$ is a smooth function defined by
	$${\displaystyle \int _{\mathbb {R} ^{d}}{f}(y)g(x-y)\,dy=(f*g)(x)\in C^{\infty }(\mathbb {R} ^{d}).} $$ \label{prop:29}
\end{propo}
\begin{propo} \label{prop:2}
	$${\frac {\partial }{\partial x_{i}}}(f*g)={\frac {\partial f}{\partial x_{i}}}*g=f*{\frac {\partial g}{\partial x_{i}}}.$$
\end{propo}

\subsection{Baire's thm}

\begin{definition} Let $A$ be a subset of the metric space $(X,d)$. $A$ is said to be \emph{nowhere dense} if for every (nonempty) open subset $U\subseteq X$, the intersection $U\cap\overline{A}$ is not dense in $U$, meaning that $U$ contains a point that is not in the closure of $A$.
\end{definition}

\begin{definition}
	A set it is said to be category $I$ if it can be written as a countable union of nowhere-dense sets. Otherwise it is said to be of \emph{category $II$}
\end{definition}
\begin{theorem} (Baire's Category Theorem) \label{baire}
	Any complete metric space is of category $II$.
\end{theorem}



\end{document}