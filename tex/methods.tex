\documentclass[../main.tex]{subfiles}
\begin{document}
\chapter{Multilayer Feedforward Networks} \label{ch:methods}
\section{Function Approximation}
\noindent In this paper we take $C(\mathbb{R}^n)$ to be the family of "real world" functions that one may wish to approximate. If we can show that a given set of functions $F$ is dense in $C(\mathbb{R}^n)$, we can conclude that for every continuous function $g \in C(\mathbb{R}^n) $ and each compact set $K \subset \mathbb{R}^n$, there is a function $f \in F$ such that $f$ is a good approximation to $g$ on K. 

\begin{definition}\label{thm:first}
	A \emph{metric} (or \emph{distance}) on a set $X$ is a function $d:X\times X\rightarrow \mathbb{R} $ such that forall $s,t\in X$ the following properties are satisfied:
	\begin{enumerate}
		\item $d(s,t) \geq 0$ and $ \, d(s,t)=0$ if and only if $ s=t$.
		\item $d(s,t)=d(t,s)$.
		\item $d(s,t)\leq d(s,u)+d(u,t)\quad$(\emph{triangular inequality}).
	\end{enumerate}
A \emph{metric space} is a pair $(X,d)$, where $X$ is a set and $d$ is a distance in $X$.
\end{definition}
\noindent  If we take $X$ to be a set of functions, the metric $d(f,g)$ will enable us to measure the distance between functions $f,g \in X$.

\begin{definition}
	Let $f,g$ be real-valued functions with compact support. We define the \emph{convolution} of $f$ with $g$ as $$(f\ast g)(x)=\int f(x-t)g(t) \, dt$$
\end{definition}

\section{Lebesgue measure}
\noindent An essential part of measure theory is the calculation of lengths, areas, volumes, etc. We are already familiar with the Riemann integral, and now we aim to introduce a more general integral, the Lebesgue integral, that comes from the Lebesgue measure. 

\begin{definition} A box in $\mathbb{R}^d$ is a set of the form 
	$$ Q = [a_1,b_1]  \times ... \times [a_d,b_d] = \prod_{i=1}^d [a_i,b_i]$$
	The volume of the box is $$ vol(Q)= (b_1,a_1) ... (b_d-a_d)= \prod_{i=1}^d (b_i-a_i)$$
	The \emph{exterior measure} (or outer measure) of a set $E\subseteq \mathbb{R}^d$ is $$ |E|^* = \inf\{\sum_k vol(Q_k)\} $$ 
	where the infimum is taken over all finite or countable collection of boxes $Q_k$ such that $E \subseteq \cup_k Q_k$
\end{definition}

\begin{definition}
A set $E\subseteq \mathbb{R}^n $ is \emph{Lebesgue mesurable} (or mesurable) if $\forall \epsilon >0$, there exist $U$ open set such that $E \subseteq U$ and $|U\setminus E|^* < \epsilon $
\end{definition}

\begin{definition}
	A function $u$ defined almost everywhere on a measurable set $\Omega \in \mathbb{R}^n$ is said to be \emph{essentially bounded} on $\Omega$ if $|u(x)|$ is bounded almost everywhere on $\Omega$. We denote $u\in L^{\infty}(\Omega)$ with the norm $$\|u\|_{L^{\infty}(\Omega)}= inf(\lambda | \{ x : |u(x)| \geq \lambda \} = 0 ) = ess \sup_{x\in \Omega} |u(x)|$$

\end{definition}

\noindent We have that $L^{\infty}(\mathbb{R})$ is the space of essentially bounded functions.
\\ \\ 
Examples and counterexamples of functions essentially bounded. 
\begin{itemize}
\item $f:\Omega \rightarrow $
\end{itemize}

\begin{definition}
	A function u defined almost everywhere on a domain $\Omega$ (a domain is an open set in $\mathbb{R}^n$) is said to be\emph{ locally essentially bounded }on $\Omega$ if for every compact set $K\subset \Omega$, $u\in L^{\infty}(K)$. We denote $u\in L_{loc}^{\infty}(K)$
\end{definition}

\begin{definition}We say that a set of functions $F\subset L_{loc}^{\infty}(\mathbb{R})$ is \emph{dense} in $C(\mathbb{R}^n)$ if for every function $g\in C(\mathbb{R}^n)$ and for every compact $K\subset \mathbb{R}^n$, there exist a sequence of functions $f_j\in F$ such that $$\lim_{j\rightarrow\infty} \|g-f_j\|_{L^\infty(K)}=0.$$ 
\end{definition}

\section{Results}
\begin{definition} Let $\mathcal{M}$ denote the set of functions which are in $L_{loc}^{\infty}(\mathbb{R})$ and have the following property. The closure of the set of points of discontinuity of any function in $\mathcal{M}$ is of zero Lebesgue measure. 
\begin{proposition}
(This implies that) for any $\sigma \in$ $\mathcal{M}$, interval $[a,b] .$ and $\delta >0$, there exists a finite number of open intervals, the union of which we denote by U, of measure $\delta$, such that $\sigma$ is uniformly continuous on $[a,b]/U$. 
\end{proposition}
\end{definition}

\section{Neural Networks}
intro \\ \\ 
The general architecture of a multilayer feedforward network, MFN, consist of: input layer: n-input units,  one/more hidden layers : intermediate processing units, output layer: m output-units
\begin{definition} (Multilayer feedforward networks) The function that a MFN compute is: 
	$$f(x)=\sum_{j=1}^k \beta_j \cdot \sigma(w_j \cdot x - \theta_j)$$
	where $x \in \mathbb{R}^n$ is the input vector, $k \in \mathbb{N}$ is the number of processing units in the hidden layer, $w_j \in \mathbb{R}^n$ is the weight vector that connects the input to processing unit $j$ in the hidden layer, $\sigma : \mathbb{R} \rightarrow \mathbb{R}$ is an activation function applied element-wise to the vector $w_j^T x - \theta_j$, where $\theta_j \in \mathbb{R}$ is the threshold (or bias) associated with processing unit $j$ in the hidden layer, and $\beta_j \in \mathbb{R}$ is the weight that connects processing unit $j$ in the hidden layer to the output of the network.

\end{definition}

	\begin{definition} 
	$ \mathcal{C}^\infty_0$ functions $\mathcal{C}^\infty$ with compact support.  
\end{definition}

\section{Theorem}
\begin{theorem} Let $ \sigma \in \text{M} $. Set
	$$ \Sigma_n = span\{\sigma(w\cdot x + \theta) : w\in \mathbb{R}^n, \theta \in \mathbb{R} \}$$
	Then $\Sigma_n$ is dense in $\mathcal{C}(\mathbb{R}^n)$ if and only if $\sigma$ is not an algebraic polynomial. 
	
\end{theorem}
\subsection{Why does not contradict the Weierstrass approximation theorem? }


\begin{theorem}(Weierstrass  approximation theorem). 
	Let $f:[a,b]\rightarrow \mathbb{R} $ be a continuous function. Then, there exists polynomials $p_n\in \mathcal{R}[x]$ such that the sequence $(p_n)$ converge uniformly to $f$ on $[a,b]$. 
\end{theorem} 

\begin{corolari}
	The set of polynomial functions $\mathcal{R}^n[x]$ is dense in the space of continuous functions on a closed and bounded interval $\mathcal{C}([a,b]^n)$. 
\end{corolari}

\noindent Com que $\sigma$ té grau fix "k", aleshores $\sigma (wx+\theta) $ grau com a molt k. Per tant el conjunt $\sigma_n$ és un espai vectorial finit i no pot ser dens. 


\end{document}