\documentclass[../main.tex]{subfiles}
\begin{document}
	\chapter{About the theorem} \label{ch:function}


\section{Results}


\begin{definition} The set $L^{p}(\mu)$ contains all mesurable functions $f$ such that: 
	$$ \|f\|_{L^p}(\mu) = \Big( \int_{R^n} |f(x)|^p d\mu(x)\Big)^{1/p} < \infty $$
	
\end{definition}

\begin{propo}
	Let $\mu $ be a non-negative finite measure on $\mathbb{R}$ with compact support, absolutely continous with respect to Lebesgue measure. Then $\Sigma_n$ is dense in $L_p(\mu)$ , $1\leq p < \infty$, if and only of, $\sigma$ is not a polynomial.
\end{propo}

\begin{propo}
	If $\sigma\in M$ is not a polynomial (a.e) then, $$ \Sigma_n(\mathcal{A})= span \{ \sigma(\lambda w \cdot x + \theta) : \lambda, \theta \in \mathbb{R}, w \in \mathcal{A} \}$$
	is dense in $\mathcal{C}(\mathbb{R}^n)$ for some $\mathcal{A}\subset \mathbb{R}^n$ if and only if there does not exist a nontrivival polynomial vanishing on $\mathcal{A}$.
\end{propo}

\begin{remark}
	The theorem only requires for the activation function to be nonpolynomial, we dont need continuity on sigma. For example, let $\sigma$ be continuous with a jump discontinuity at 0 such that: $$\lim_{x\to 0^-} \sigma(x)=0 \qquad \lim_{x \to 0^+} \sigma(x) =1$$
	Given $f \in \mathcal{C}(\mathbb{R})$ and $K \subset \mathbb{R}$ compact, letting $w \rightarrow 0$ in $\sigma(wx)$ the function
	
	\[
	h(x) = \begin{cases} 
		0 & \text{for } x < 0 \\
		1 & \text{for } x > 0 
	\end{cases}
	\]
	$h\in \overline{\Sigma_1}$. \\ \\ 
	Linear combinations of h and its translates can unifformly approximate any continuous function on any finite interval (and thus an), compact subset of R).
	
\end{remark}

\end{document}