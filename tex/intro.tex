\documentclass[../main.tex]{subfiles}

\begin{document}

    \chapter{Introduction} \label{ch:intro}
    
    \say{Experimental science was born. But experiment is a tool. The aim remains: to understand the world. To restrict quantum mechanics to be exclusively about piddling laboratory operations is to betray the great enterprise. A serious formulation will not exclude the big world outside the laboratory.}{John S. Bell}{Against Measurement}


\noindent Early computers were used to perform exact computations with high accuracy and efficiency. Back in 1945, one of the first electronic computer was invented for ballistic calculations during World War II. Computers seemed to be limited to these exact computation tasks. However, over time, researchers started pushing the boundaries of what computers can do, eventually leading to the development of what we call now Artificial Intelligence. AI  seeks to make computers do the sorts of things that minds can do.

    
    
    Some of these (e.g. reasoning) are normally described as “intelligent.” Others (e.g. vision) aren’t. But all involve psychological
    skills—such as perception, association, prediction, planning, motor
    control—that enable humans and animals to attain their goals. \\ \\
    3. on les matematiques prenen lloc ? pq son importants .  Can we suggest conjectures, relationships , theorems between fields ??? using ml as a tool to see unexpected relationships. 
    
    ML might become a bycicle for the mind !!
    
    MATHS USING MCH LEARNING <--> ML USING MATHS  
    
   

\end{document}