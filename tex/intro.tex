\documentclass[../main.tex]{subfiles}

\begin{document}

    \chapter{Introduction} \label{ch:intro}
    
    \say{Computers are like a bicycle for our minds.}{Steve Jobs}{Michael Lawrence Films}


\noindent Our brain is constantly classifying and recognizing. For instance, when we spot a dog on the street, one easy classification we can make is  $\{ \text{dog, not dog} \}$, which is probably too easy for our brain—it's almost instantaneous.  However, things get a bit more complex when we read the teacher's whiteboard. What happens when we encounter a symbol that confuses us because it resembles another?
We can interpret the mathematics behind this reasoning as the brain seeking/creating a function that provides us with the certainty of recognizing that particular letter. Eventually, we reach a point where we feel confident enough to write it down in our notes. \\ \\
Artificial intelligence aims to replicate the remarkable capabilities of our brains. It seeks to develop computational models and algorithms that can perform tasks such as classification, recognition, and decision-making with a level of accuracy and efficiency comparable to human intelligence. When AI first emerged, one of the initial challenges was hand-written digit recognition, exemplified by the MNIST digits dataset. This dataset comprises 60,000 examples of handwritten digits from 0 to 9. To enable a machine learning model to recognize these digits, it must effectively map each image to its corresponding number.
This problem naturally aligns with a mathematician's perspective of function learning, where the goal is to approximate a function based on a given dataset consisting of points in space.
\\ \\ 
Neural Networks are a key approach used in artificial intelligence to tackle such problems.  The theory of function approximation through neural networks has a long history dating back to the work by McCulloch and Pitts \\ \\
This Bachelor's thesis aims to dig into the mathematical foundations of machine learning, 
Our main ... is to demonstrate that the "real-world" functions we seek to approximate can be effectively approximated by a specific type of functions. \\ \\ 


\end{document}